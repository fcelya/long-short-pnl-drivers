\section{Conclusion}

Although pairs-trading strategies have a long history, both in academic literature and in practice, most research has been centered on developing novel strategies and improving their performance through different adaptations. There is still work to be done regarding the explainability of its returns and the driver analysis for explanatory and practical purposes. 
In this work, the already developed implementation of a cointegration based pairs-trading strategy has been modified in order to make it practical to apply it to a larger universe of assets and to improve its performance through some adaptations. During the implemented backtest more information is stored regarding the returns of the long and short leg of the strategy and the number of open positions at each timestep. A comparable benchmark based on Johansen cointegration is developed and implemented. The results of the complete strategy are examined first from a performance perspective comparing it to the market and the developed benchmark. This performance analysis is enhanced by also studying the performance of the long and short legs of the strategy independently, examining what each side contributes to the overall strategy. 
After the performance is studied, the main drivers of this performance are analysed. A factor model is developed and implemented and the returns of the strategy are used to obtain the factor loadings for said model, shedding some light on which of the factors are responsible for the studied results. After a low $R^2$ for the model fit and a significant $\alpha$ of unexplained outperformance is obtained, this $\alpha$ is resampled quarterly and a second regression on company specific disclosed data relating to valuation and profitability is performed. The results obtained through the fitting of this secondary model shed some further light on which specific financial factors contribute as drivers of the performance of the strategy. 

The implemented strategy has been shown to clearly outperform both the market and the comparable benchmark on a risk adjusted basis on pretty much all performance metrics, from financial ratios to statistic measures. It has also been shown to have a low negative beta with respect to the market and thus provide a very useful point of difersification for any market participant. Both the long and short leg of the strategy have been shown to have a very similar return profile to the overall strategy, with the long leg having higher risk and higher return performance, and the short leg having lower risk and lower returns. The combination of the two has also been demonstrated to have a better risk adjusted performance than any of the individual halves. It has also been shown how the performance seems to be unrelated to the number of open positions. 
The primary regression with the five factor model has been shown to be robust and all coefficients have been found to be highly significant. The strategy has been shown to, in addition to the negative exposure to market already mentioned, be mainly driven by the size factor and the Idiosyncratic Volatility Spread, both being significant positive drivers of the PnL. There has also been shown to be a smaller negative exposure to the momentum and value premium factors. The outperformance of this strategy relative to this model has been shown to be singificant at an annualized value of $\alpha=10.08\%$. 
The secondary regression has shown that some of the factors initially thought to be drivers of the strategy -- namely $GPA$, $EVEBIT$ and $PER$ -- were in fact not significant. The significance of the rest was shown with a more robust model reached through iterative elimination of non significant factors. The strategy's performance was shown to be favoured by exposure to companies with high market capitalization, highly profitable as measured by the Returns on Capital, and with high valuations as measured by the Enterprise Value to EBITDA and Book-to-Market ratios. 

There are still many interesting avenues for future research, both in the analysis of the performance of the strategy and the study of its main drivers. 
As for the implementation, a more realistic backtest could be implemented by including transaction costs, lending costs of the short positions, slippage, risk of margin calls, etc. This would lead to a more realistic performance backtest. Also, more time periods and universes of assets could be analysed to ensure the robustness of the outperformance of the strategy.
Regarding the driver analysis, there is room for more explainability through the inclusion of factors different to those that have been included here, with sector exposure being a promising start. It should also be explored whether a selective filtering of the assets based on its positive drivers leads to better risk adjusted performance. It will be interesting to see how more analysis of the key drivers of pairs-trading strategies and on a deeper and more disaggregated way could lead to better performance in practice.