\section{Introduction}
Long-short trading strategies, particularly pairs trading strategies, have been an enduring tool in quantitative finance due to their ability to generate market-neutral returns. There have been many discussions on how to optimize each part of a pairs trading strategy and maximize its performance. However, little attention has been given to what the underlying factors of such performance are. Exploring these factors can help derive insight that will improve the understanding of these strategies and maximize their risk adjusted returns. 

This work first provides some useful background regarding long-short and pairs trading strategies. Then, the methodology is outlined explaining every part of the implemented study. 
Namely, a pairs-trading implementation proposed by \cite{gallego_2023} is reviewed and adapted, from a fundamental point of view to add some changes to the way the strategy proposed by \cite{ioannis_2023} was implemented and from a technical point of view, so it is able to successfully perform a backtest on a much larger universe of assets. Then, this strategy is run on an extended universe of assets and its performance is analyzed by its own merit and in comparison to a benchmark strategy. The performance is then dissected into the long and short positions sepparately and each leg is further analyzed. 
Then, a primary factor model is implemented and the returns of the strategy are analyzed through this model.  
Lastly, the alpha obtained by the strategy according to said model is further analyzed through a secondary fundamental factor model in order to further extract all possible key drivers of the returns of the strategy. 
Out of all the types of PnL drivers outlined in section \nameref{sec:classes-pnl-drivers}, only market exposure, some factor sensitivities and alpha generation are analyzed. Transaction costs are considered negligible due to the daily frequency of transactions and the small impact compared with the other factors, while slippage and liquidity and market impact have also been ignored due to the high trading volume of the selected universe of assets, the S\&P500. 
After the methodology, a brief description of the used data and the technicalities of the implementation are given. Finally, the results are presented and the conclusion is layed out. 

These results demonstrate that the implemented strategy has a great risk adjusted performance and can effectively be used as a market hedge. More importantly, the main PnL drivers are shown to be the size factor and the Idiosyncratic Equity Volatility Premium, both being significant positive drivers of the PnL. There has also been shown to be a smaller negative exposure to momentum and value premium factors. Through the second factor regression the excess unexplained performance has been shown to be driven mainly by exposure to companies with high market capitalization, highly profitable as measured by the Return on Capital and with high EV to EBITDA and low BTM ratios.